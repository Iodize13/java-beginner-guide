% Created 2025-01-08 Wed 18:06
% Intended LaTeX compiler: pdflatex
\documentclass[11pt]{article}
\usepackage[utf8]{inputenc}
\usepackage[T1]{fontenc}
\usepackage{graphicx}
\usepackage{longtable}
\usepackage{wrapfig}
\usepackage{rotating}
\usepackage[normalem]{ulem}
\usepackage{amsmath}
\usepackage{amssymb}
\usepackage{capt-of}
\usepackage{hyperref}
\date{\today}
\title{}
\hypersetup{
 pdfauthor={},
 pdftitle={},
 pdfkeywords={},
 pdfsubject={},
 pdfcreator={Emacs 29.4 (Org mode 9.7.11)}, 
 pdflang={English}}
\begin{document}

\tableofcontents

\section{\href{./index.org}{\textbf{Java}}}
\label{sec:org1322d8f}
\section{สรุปเนื้อหาเรื่อง Arrays (พร้อมตัวอย่างโค้ด)}
\label{sec:org4c7f192}
ความยาก: \textbf{**}

\textbf{เนื้อหาต่อจากนี้ถูกสร้างขึ้นโดย AI โดยใช้ข้อมูลจาก Presentation Slides เรื่อง Arrays}

\noindent\rule{\textwidth}{0.5pt}
\subsection{การสร้างและใช้งาน Array}
\label{sec:org2682746}
\begin{verbatim}
// สร้าง Array เก็บจำนวนเต็ม 5 ค่า
int[] numbers = new int[5];

// กำหนดค่าใน Array
numbers[0] = 10;
numbers[1] = 20;
numbers[2] = 30;
numbers[3] = 40;
numbers[4] = 50;

// แสดงผลลัพธ์
for (int i = 0; i < numbers.length; i++) {
    System.out.println("Element " + i + ": " + numbers[i]);
}
\end{verbatim}

\noindent\rule{\textwidth}{0.5pt}
\subsection{Array Initialization (การกำหนดค่าเริ่มต้น)}
\label{sec:org6b70f35}
\begin{verbatim}
// สร้างและกำหนดค่าเริ่มต้นใน Array
int[] list = {1, 2, 3, 4, 5};

// แสดงผลลัพธ์
for (int num : list) {
    System.out.println(num);
}
\end{verbatim}

\noindent\rule{\textwidth}{0.5pt}
\subsection{Frequents Errors About Array \textbf{*}}
\label{sec:orgcbc250f}
หากพยายามเข้าถึง index ที่ไม่มีใน array จะเกิดข้อผิดพลาด
\texttt{ArrayIndexOutOfBoundsException}

\begin{verbatim}
int[] list = {1, 2, 3};
System.out.println(list[3]);  // จะเกิด ArrayIndexOutOfBoundsException
\end{verbatim}

\noindent\rule{\textwidth}{0.5pt}
\subsection{Property \texttt{length} ใน Array หลายมิติ}
\label{sec:org915290c}
\begin{itemize}
\item \texttt{A.length} ให้จำนวนแถวของ array
\item \texttt{A[0].length} ให้จำนวนคอลัมน์ในแถวแรก
\end{itemize}

\begin{verbatim}
int[][] A = {
    {1, 2, 3, 4},
    {5, 6, 7, 8},
    {9, 10, 11, 12}
};

System.out.println("Number of rows: " + A.length);        // ผลลัพธ์: 3
System.out.println("Number of columns in row 0: " + A[0].length);  // ผลลัพธ์: 4
\end{verbatim}

\noindent\rule{\textwidth}{0.5pt}
\subsection{Multi-dimensional Arrays (Array หลายมิติ) *}
\label{sec:orge80bad3}
\begin{verbatim}
// สร้าง 2D Array (3 แถว 4 คอลัมน์)
int[][] matrix = {
    {1, 2, 3, 4},
    {5, 6, 7, 8},
    {9, 10, 11, 12}
};

// แสดงผลลัพธ์
for (int i = 0; i < matrix.length; i++) {
    for (int j = 0; j < matrix[i].length; j++) {
        System.out.print(matrix[i][j] + " ");
    }
    System.out.println();
}
\end{verbatim}

\noindent\rule{\textwidth}{0.5pt}
\subsection{การค้นหาค่าใน Array (Linear Search) *}
\label{sec:org74b0cb3}
\begin{verbatim}
public class LinearSearch {
    public static int search(int[] array, int target) {
        for (int i = 0; i < array.length; i++) {
            if (array[i] == target) {
                return i;  // คืนค่า index ที่เจอ
            }
        }
        return -1;  // คืนค่า -1 ถ้าไม่เจอ
    }

    public static void main(String[] args) {
        int[] numbers = {10, 20, 30, 40, 50};
        int index = search(numbers, 30);
        System.out.println("Found at index: " + index);  // ผลลัพธ์: Found at index: 2
    }
}
\end{verbatim}

\noindent\rule{\textwidth}{0.5pt}
\subsection{การคัดลอกค่าใน Array (Array Copy) *}
\label{sec:org5bb3115}
\begin{itemize}
\item \textbf{วิธีที่ 1: การคัดลอกแบบ Manual}
\end{itemize}
\begin{verbatim}
int[] source = {1, 2, 3, 4, 5};
int[] destination = new int[source.length];

// คัดลอกค่า
for (int i = 0; i < source.length; i++) {
    destination[i] = source[i];
}

// แสดงผลลัพธ์
for (int num : destination) {
    System.out.println(num);
}
\end{verbatim}

\begin{itemize}
\item \textbf{วิธีที่ 2: การคัดลอกด้วย System.arraycopy()}
\end{itemize}
ความยาก: *
\begin{verbatim}
int[] source = {1, 2, 3, 4, 5};
int[] destination = new int[source.length];

// ใช้ System.arraycopy() ในการคัดลอก
System.arraycopy(source, 0, destination, 0, source.length);

// แสดงผลลัพธ์
for (int num : destination) {
    System.out.println(num);
}
\end{verbatim}

\noindent\rule{\textwidth}{0.5pt}
\subsection{การเรียงลำดับค่าใน Array (Sorting) \textbf{**}}
\label{sec:org8fe34b1}
\begin{itemize}
\item การเรียงลำดับค่าใน Array ด้วย Selection Sort
\end{itemize}
ความยาก: \textbf{**}
\begin{verbatim}
public class SelectionSort {
    public static void sort(int[] array) {
        for (int i = 0; i < array.length - 1; i++) {
            int minIndex = i;
            for (int j = i + 1; j < array.length; j++) {
                if (array[j] < array[minIndex]) {
                    minIndex = j;
                }
            }
            int temp = array[minIndex];
            array[minIndex] = array[i];
            array[i] = temp;
        }
    }

    public static void main(String[] args) {
        int[] numbers = {64, 25, 12, 22, 11};
        sort(numbers);
        for (int num : numbers) {
            System.out.print(num + " ");  // ผลลัพธ์: 11 12 22 25 64
        }
    }
}
\end{verbatim}
\begin{itemize}
\item \textbf{การเรียงลำดับด้วย Insertion Sort}
\end{itemize}
ความยาก: \textbf{**}
\begin{verbatim}
public class InsertionSort {
    public static void sort(int[] array) {
        for (int i = 1; i < array.length; i++) {
            int key = array[i];
            int j = i - 1;

            while (j >= 0 && array[j] > key) {
                array[j + 1] = array[j];
                j--;
            }
            array[j + 1] = key;
        }
    }

    public static void main(String[] args) {
        int[] numbers = {5, 2, 9, 1, 5, 6};
        sort(numbers);

        for (int num : numbers) {
            System.out.print(num + " ");  // ผลลัพธ์: 1 2 5 5 6 9
        }
    }
}
\end{verbatim}

\noindent\rule{\textwidth}{0.5pt}
\subsection{การค้นหาด้วย Binary Search (Binary Search) \textbf{*}}
\label{sec:orgcd2390e}
\begin{verbatim}
public class BinarySearch {
    public static int binarySearch(int[] array, int target) {
        int low = 0;
        int high = array.length - 1;

        while (low <= high) {
            int mid = (low + high) / 2;

            if (array[mid] == target) {
                return mid;  // คืนค่า index ที่เจอ
            } else if (array[mid] < target) {
                low = mid + 1;
            } else {
                high = mid - 1;
            }
        }
        return -1;  // คืนค่า -1 ถ้าไม่เจอ
    }

    public static void main(String[] args) {
        int[] numbers = {1, 2, 3, 4, 5, 6, 7, 8, 9};
        int index = binarySearch(numbers, 5);
        System.out.println("Found at index: " + index);  // ผลลัพธ์: Found at index: 4
    }
}
\end{verbatim}

\noindent\rule{\textwidth}{0.5pt}
\subsection{การใช้ Arrays Utility ใน Java *}
\label{sec:orgd370dc5}
\begin{verbatim}
import java.util.Arrays;

public class ArrayUtilityExample {
    public static void main(String[] args) {
        int[] numbers = {5, 2, 9, 1, 5, 6};

        // เรียงลำดับ Array
        Arrays.sort(numbers);

        // แสดงผลลัพธ์หลังเรียงลำดับ
        System.out.println(Arrays.toString(numbers));  // ผลลัพธ์: [1, 2, 5, 5, 6, 9]

        // ค้นหาด้วย Binary Search
        int index = Arrays.binarySearch(numbers, 5);
        System.out.println("Found at index: " + index);  // ผลลัพธ์: 2
    }
}
\end{verbatim}

\noindent\rule{\textwidth}{0.5pt}
\end{document}
