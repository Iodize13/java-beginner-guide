% Created 2025-01-08 Wed 18:06
% Intended LaTeX compiler: pdflatex
\documentclass[11pt]{article}
\usepackage[utf8]{inputenc}
\usepackage[T1]{fontenc}
\usepackage{graphicx}
\usepackage{longtable}
\usepackage{wrapfig}
\usepackage{rotating}
\usepackage[normalem]{ulem}
\usepackage{amsmath}
\usepackage{amssymb}
\usepackage{capt-of}
\usepackage{hyperref}
\date{\today}
\title{}
\hypersetup{
 pdfauthor={},
 pdftitle={},
 pdfkeywords={},
 pdfsubject={},
 pdfcreator={Emacs 29.4 (Org mode 9.7.11)}, 
 pdflang={English}}
\begin{document}

\tableofcontents

\section{\href{./index.org}{\textbf{Java}}}
\label{sec:orgd7e1155}
\section{สรุปเนื้อหาเรื่อง Control Statements (ภาษาไทย พร้อมตัวอย่างโค้ด)}
\label{sec:org951248f}
ความยาก: *

\textbf{เนื้อหาต่อจากนี้ถูกสร้างขึ้นโดย AI โดยใช้ข้อมูลจาก Presentation Slides เรื่อง Control Statements}

\noindent\rule{\textwidth}{0.5pt}
\subsection{if Statement}
\label{sec:org04a4692}
\begin{verbatim}
int number = 10;
if (number > 0) {
    System.out.println("The number is positive.");  // ผลลัพธ์: The number is positive.
}
\end{verbatim}

\noindent\rule{\textwidth}{0.5pt}
\subsection{if\ldots{}else Statement}
\label{sec:orgede11d1}
\begin{verbatim}
int number = -5;
if (number > 0) {
    System.out.println("The number is positive.");
} else {
    System.out.println("The number is negative.");  // ผลลัพธ์: The number is negative.
}
\end{verbatim}

\noindent\rule{\textwidth}{0.5pt}
\subsection{Nested if Statement}
\label{sec:org850fcbb}
\begin{verbatim}
int score = 85;

if (score >= 90) {
    System.out.println("Grade: A");
} else if (score >= 80) {
    System.out.println("Grade: B");  // ผลลัพธ์: Grade: B
} else if (score >= 70) {
    System.out.println("Grade: C");
} else {
    System.out.println("Grade: F");
}
\end{verbatim}

\noindent\rule{\textwidth}{0.5pt}
\subsection{switch Statement **}
\label{sec:orgad536ac}
\begin{verbatim}
int day = 3;

switch (day) {
    case 1:
        System.out.println("Monday");
        break;
    case 2:
        System.out.println("Tuesday");
        break;
    case 3:
        System.out.println("Wednesday");  // ผลลัพธ์: Wednesday
        break;
    default:
        System.out.println("Invalid day");
}
\end{verbatim}

\noindent\rule{\textwidth}{0.5pt}
\subsection{while Loop}
\label{sec:orgd1f3c6c}
\begin{verbatim}
int i = 0;
while (i < 5) {
    System.out.println("Count: " + i);
    i++;
}
// ผลลัพธ์:
// Count: 0
// Count: 1
// Count: 2
// Count: 3
// Count: 4
\end{verbatim}

\noindent\rule{\textwidth}{0.5pt}
\subsection{do\ldots{}while Loop}
\label{sec:org813210a}
\begin{verbatim}
int i = 0;
do {
    System.out.println("Count: " + i);
    i++;
} while (i < 5);
// ผลลัพธ์เหมือนกับ while loop
\end{verbatim}

\noindent\rule{\textwidth}{0.5pt}
\subsection{for Loop}
\label{sec:org6f2c711}
\begin{verbatim}
for (int i = 1; i <= 5; i++) {
    System.out.println("Iteration: " + i);
}
// ผลลัพธ์:
// Iteration: 1
// Iteration: 2
// Iteration: 3
// Iteration: 4
// Iteration: 5
\end{verbatim}

\noindent\rule{\textwidth}{0.5pt}
\subsection{Nested for Loop}
\label{sec:orgd476b72}
\begin{verbatim}
for (int i = 1; i <= 2; i++) {
    for (int j = 1; j <= 4; j++) {
        System.out.print(j + " ");
    }
    System.out.println();
}
// ผลลัพธ์:
// 1 2 3 4
// 1 2 3 4
\end{verbatim}

\noindent\rule{\textwidth}{0.5pt}
\subsection{break Statement *}
\label{sec:orgfc4dee0}
\noindent\rule{\textwidth}{0.5pt}
\begin{verbatim}
for (int i = 1; i <= 5; i++) {
    if (i == 3) {
        break;
    }
    System.out.println("Iteration: " + i);
}
// ผลลัพธ์:
// Iteration: 1
// Iteration: 2
\end{verbatim}

\noindent\rule{\textwidth}{0.5pt}
\subsection{continue Statement *}
\label{sec:org8563585}
\begin{verbatim}
for (int i = 1; i <= 5; i++) {
    if (i == 3) {
        continue;
    }
    System.out.println("Iteration: " + i);
}
// ผลลัพธ์:
// Iteration: 1
// Iteration: 2
// Iteration: 4
// Iteration: 5
\end{verbatim}

\noindent\rule{\textwidth}{0.5pt}
\subsection{การใช้ Scanner รับค่าจากผู้ใช้ **}
\label{sec:orge5f022b}
\begin{verbatim}
import java.util.Scanner;

public class UserInput {
    public static void main(String[] args) {
        Scanner scanner = new Scanner(System.in);

        System.out.print("Enter your age: ");
        int age = scanner.nextInt();

        if (age >= 18) {
            System.out.println("You are an adult.");
        } else {
            System.out.println("You are not an adult.");
        }

        scanner.close();
    }
}
\end{verbatim}
\end{document}
