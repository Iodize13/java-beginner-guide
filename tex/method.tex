% Created 2025-01-08 Wed 18:06
% Intended LaTeX compiler: pdflatex
\documentclass[11pt]{article}
\usepackage[utf8]{inputenc}
\usepackage[T1]{fontenc}
\usepackage{graphicx}
\usepackage{longtable}
\usepackage{wrapfig}
\usepackage{rotating}
\usepackage[normalem]{ulem}
\usepackage{amsmath}
\usepackage{amssymb}
\usepackage{capt-of}
\usepackage{hyperref}
\date{\today}
\title{}
\hypersetup{
 pdfauthor={},
 pdftitle={},
 pdfkeywords={},
 pdfsubject={},
 pdfcreator={Emacs 29.4 (Org mode 9.7.11)}, 
 pdflang={English}}
\begin{document}

\tableofcontents

\section{\href{./index.org}{\textbf{Java}}}
\label{sec:org24f807e}
\section{สรุปเนื้อหาเรื่อง Methods (พร้อมตัวอย่างโค้ด)}
\label{sec:org434b6b8}
ความยาก: **

\textbf{เนื้อหาต่อจากนี้ถูกสร้างขึ้นโดย AI โดยใช้ข้อมูลจาก Presentation Slides เรื่อง Methods}

\noindent\rule{\textwidth}{0.5pt}
\subsection{ตัวอย่างโครงสร้าง Method พื้นฐาน}
\label{sec:org275f874}
\begin{verbatim}
public class MethodsExample {
    // Method ชื่อ add รับพารามิเตอร์สองตัวและคืนค่าผลรวม
    public static int add(int a, int b) {
        return a + b;
    }

    public static void main(String[] args) {
        int result = add(5, 10);
        System.out.println("Result: " + result);  // ผลลัพธ์: 15
    }
}
\end{verbatim}

\noindent\rule{\textwidth}{0.5pt}
\subsection{ตัวอย่างการประกาศและนิยาม Method}
\label{sec:org04eca66}
\begin{verbatim}
// การประกาศ Method
static int max(int num1, int num2);

// การนิยาม Method
static int max(int num1, int num2) {
    if (num1 > num2) {
        return num1;
    } else {
        return num2;
    }
}
\end{verbatim}

\noindent\rule{\textwidth}{0.5pt}
\subsection{การเรียกใช้งาน Method (Method Call)}
\label{sec:org9c314df}
\begin{verbatim}
public static void main(String[] args) {
    int i = 5;
    int j = 2;
    int result = max(i, j);
    System.out.println("Max value is: " + result);  // ผลลัพธ์: Max value is: 5
}
\end{verbatim}

\noindent\rule{\textwidth}{0.5pt}
\subsection{การส่งค่าพารามิเตอร์ (Passing Parameters)}
\label{sec:org0b8197f}
\begin{verbatim}
// Method สลับค่าตัวแปรสองตัว
static void swap(int a, int b) {
    int temp = a;
    a = b;
    b = temp;
    System.out.println("Inside swap method: a = " + a + ", b = " + b);
}

public static void main(String[] args) {
    int m = 2;
    int n = 3;
    System.out.println("Before swap: m = " + m + ", n = " + n);
    swap(m, n);
    System.out.println("After swap: m = " + m + ", n = " + n);  // ค่าไม่เปลี่ยน
}
\end{verbatim}

\noindent\rule{\textwidth}{0.5pt}
\subsection{การ Overloading Method **}
\label{sec:orgdfca0f6}
\begin{verbatim}
// การ Overloading Method ด้วยชื่อเดียวกันแต่ต่างกันที่ชนิดพารามิเตอร์
static int max(int num1, int num2) {
    return (num1 > num2) ? num1 : num2;
}

static double max(double num1, double num2) {
    return (num1 > num2) ? num1 : num2;
}

public static void main(String[] args) {
    System.out.println(max(5, 10));        // ผลลัพธ์: 10
    System.out.println(max(5.5, 10.1));    // ผลลัพธ์: 10.1
}
\end{verbatim}

\noindent\rule{\textwidth}{0.5pt}
\subsection{Method ที่เรียกตัวเอง (Recursive Method) \textbf{*}}
\label{sec:orgdeda434}
\begin{verbatim}
// การคำนวณ Factorial ด้วยการใช้ Recursive Method
static long factorial(int n) {
    if (n == 0) { // เงื่อนไขหยุด
        return 1;
    } else {
        return n * factorial(n - 1);
    }
}

public static void main(String[] args) {
    int number = 5;
    System.out.println("Factorial of " + number + " is " + factorial(number));  // ผลลัพธ์: 120
}
\end{verbatim}
\subsubsection{Recursive Method Stopping Condition}
\label{sec:orge769fac}
ถ้า Recursive Method ไม่มีเงื่อนไขหยุด จะทำให้เกิด \textbf{StackOverflowError}

\noindent\rule{\textwidth}{0.5pt}
\subsection{การใช้งาน Method จากแพ็กเกจมาตรฐาน (API) *}
\label{sec:org25e10c9}
\begin{verbatim}
import java.util.Scanner;

public class InputExample {
    public static void main(String[] args) {
        Scanner input = new Scanner(System.in);
        System.out.print("Enter a number: ");
        int number = input.nextInt();
        System.out.println("You entered: " + number);
        input.close();
    }
}
\end{verbatim}
\end{document}
