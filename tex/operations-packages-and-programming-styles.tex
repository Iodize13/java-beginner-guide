% Created 2025-01-08 Wed 18:06
% Intended LaTeX compiler: pdflatex
\documentclass[11pt]{article}
\usepackage[utf8]{inputenc}
\usepackage[T1]{fontenc}
\usepackage{graphicx}
\usepackage{longtable}
\usepackage{wrapfig}
\usepackage{rotating}
\usepackage[normalem]{ulem}
\usepackage{amsmath}
\usepackage{amssymb}
\usepackage{capt-of}
\usepackage{hyperref}
\date{\today}
\title{}
\hypersetup{
 pdfauthor={},
 pdftitle={},
 pdfkeywords={},
 pdfsubject={},
 pdfcreator={Emacs 29.4 (Org mode 9.7.11)}, 
 pdflang={English}}
\begin{document}

\tableofcontents

\section{\href{./index.org}{\textbf{Java}}}
\label{sec:org1c08fc1}
\section{สรุปเนื้อหาเรื่อง Operations, Packages, and Programming Styles (พร้อมตัวอย่างโค้ด)}
\label{sec:orgc04df1e}
\textbf{เนื้อหาต่อจากนี้ถูกสร้างขึ้นโดย AI โดยใช้ข้อมูลจาก Presentation Slides เรื่อง Operations, Packages, and Programming Styles}

\noindent\rule{\textwidth}{0.5pt}
\subsection{การดำเนินการทางคณิตศาสตร์ใน Java (Arithmetic Operations)}
\label{sec:org3599026}
\begin{itemize}
\item \textbf{ตัวอย่างการใช้งานตัวดำเนินการพื้นฐาน}
\end{itemize}
\begin{verbatim}
int a = 10;
int b = 3;

System.out.println(a + b);   // ผลลัพธ์: 13
System.out.println(a - b);   // ผลลัพธ์: 7
System.out.println(a * b);   // ผลลัพธ์: 30
System.out.println(a / b);   // ผลลัพธ์: 3
System.out.println(a % b);   // ผลลัพธ์: 1
\end{verbatim}

\noindent\rule{\textwidth}{0.5pt}
\subsection{ตัวดำเนินการแบบย่อ (Shortcut Operators)}
\label{sec:org4122d5b}
\begin{verbatim}
int num = 5;

num += 3;  // num = num + 3;
System.out.println(num);  // ผลลัพธ์: 8

num *= 2;  // num = num * 2;
System.out.println(num);  // ผลลัพธ์: 16
\end{verbatim}

\noindent\rule{\textwidth}{0.5pt}
\subsection{การเพิ่มและลดค่า (Increment \& Decrement Operators)}
\label{sec:org7df76c2}
\begin{verbatim}
int x = 1;

// Post-Increment (x++ ใช้ค่าเดิมก่อนค่อยเพิ่ม)
System.out.println(x++);  // ผลลัพธ์: 1
System.out.println(x);    // ผลลัพธ์: 2

// Pre-Increment (++x เพิ่มค่าก่อนแล้วค่อยใช้)
System.out.println(++x);  // ผลลัพธ์: 3
\end{verbatim}

\noindent\rule{\textwidth}{0.5pt}
\subsection{ตัวดำเนินการเปรียบเทียบ (Relational Operators)}
\label{sec:org0086cbb}
\begin{verbatim}
int a = 10;
int b = 5;

System.out.println(a == b);   // ผลลัพธ์: false
System.out.println(a != b);   // ผลลัพธ์: true
System.out.println(a > b);    // ผลลัพธ์: true
System.out.println(a <= b);   // ผลลัพธ์: false
\end{verbatim}

\noindent\rule{\textwidth}{0.5pt}
\subsection{ตัวดำเนินการแบบเงื่อนไข (Conditional Operator) *}
\label{sec:org4d08e4c}
\begin{verbatim}
int n = 3;
int next = (n % 2 == 0) ? (n / 2) : (3 * n + 1);
System.out.println(next);  // ผลลัพธ์: 10
\end{verbatim}

\noindent\rule{\textwidth}{0.5pt}
\subsection{การแปลงชนิดข้อมูล (Type Casting) *}
\label{sec:org8c8078b}
\begin{verbatim}
int a = 10;
double b = a;  // แปลงจาก int เป็น double อัตโนมัติ

double c = 9.99;
int d = (int) c;  // แปลงจาก double เป็น int โดยใช้การ cast
System.out.println(d);  // ผลลัพธ์: 9
\end{verbatim}

\textbf{ข้อควรระวัง}: การแปลงจาก \texttt{double} เป็น \texttt{int} จะตัดทศนิยมทิ้ง

\noindent\rule{\textwidth}{0.5pt}
\subsection{การใช้แพ็กเกจใน Java (Java Packages) *}
\label{sec:orge6ec761}
\begin{itemize}
\item \textbf{ตัวอย่างการนำเข้าแพ็กเกจ}
\end{itemize}
\begin{verbatim}
import java.util.*; //แนะนำให้ใช้ * เพราะรวมทุก packages ให้เลย

public class InputExample {
    public static void main(String[] args) {
        Scanner input = new Scanner(System.in);
        System.out.print("Enter your name: ");
        String name = input.nextLine();
        System.out.println("Hello, " + name);
        input.close();
    }
}
\end{verbatim}

\noindent\rule{\textwidth}{0.5pt}
\subsection{การจัดการเอกสารโค้ดด้วย Javadoc *}
\label{sec:org48ee709}
\begin{itemize}
\item \textbf{ตัวอย่าง Javadoc Comment}
\end{itemize}
\begin{verbatim}
/**
 * This class demonstrates how to use Javadoc.
 * @author Oh
 * @version 1.0
 */
public class JavadocExample {

    /**
     * This method adds two numbers.
     * @param a the first number
     * @param b the second number
     * @return the sum of a and b
     */
    public static int add(int a, int b) {
        return a + b;
    }

    public static void main(String[] args) {
        System.out.println("Sum: " + add(5, 10));
    }
}
\end{verbatim}

\noindent\rule{\textwidth}{0.5pt}
\end{document}
