% Created 2025-01-08 Wed 18:06
% Intended LaTeX compiler: pdflatex
\documentclass[11pt]{article}
\usepackage[utf8]{inputenc}
\usepackage[T1]{fontenc}
\usepackage{graphicx}
\usepackage{longtable}
\usepackage{wrapfig}
\usepackage{rotating}
\usepackage[normalem]{ulem}
\usepackage{amsmath}
\usepackage{amssymb}
\usepackage{capt-of}
\usepackage{hyperref}
\date{\today}
\title{}
\hypersetup{
 pdfauthor={},
 pdftitle={},
 pdfkeywords={},
 pdfsubject={},
 pdfcreator={Emacs 29.4 (Org mode 9.7.11)}, 
 pdflang={English}}
\begin{document}

\tableofcontents

\section{\href{./index.org}{\textbf{Java}}}
\label{sec:org7645b69}
\section{สรุปเนื้อหาเรื่อง Variables and Data Types (พร้อมตัวอย่างโค้ด)}
\label{sec:org04ab3fc}
ความยาก: **

\textbf{เนื้อหาต่อจากนี้ถูกสร้างขึ้นโดย AI โดยใช้ข้อมูลจาก Presentation Slides เรื่อง Variable and Data types}

\noindent\rule{\textwidth}{0.5pt}
\subsection{การเขียนโปรแกรม Java เบื้องต้น}
\label{sec:orgabdad72}
\begin{itemize}
\item \textbf{Class (คลาส)}: โครงสร้างพื้นฐานของโปรแกรม Java
ทุกโปรแกรมจะต้องเริ่มต้นด้วยคลาส

\item \textbf{Main Method (เมธอดหลัก)}: จุดเริ่มต้นของการทำงานในโปรแกรม Java

\begin{verbatim}
  public class HelloWorld {
      public static void main(String[] args) {
          System.out.println("Hello World!");
      }
  }
\end{verbatim}
\end{itemize}

\noindent\rule{\textwidth}{0.5pt}
\subsection{ตัวแปร (Variables)}
\label{sec:org6b18202}
\begin{itemize}
\item \textbf{ตัวแปร} คือ กล่องที่ใช้เก็บข้อมูลในหน่วยความจำ

\item การประกาศตัวแปรใน Java ต้องระบุ \textbf{ชนิดข้อมูล (Type)} ก่อนเสมอ เช่น \texttt{int},
\texttt{double}, \texttt{String}

\begin{verbatim}
  int age = 25;
  double salary = 50000.0;
\end{verbatim}
\end{itemize}

\noindent\rule{\textwidth}{0.5pt}
\subsection{ชนิดข้อมูลพื้นฐาน (Primitive Data Types)}
\label{sec:org790bc9b}
\begin{center}
\begin{tabular}{lrl}
\textbf{ชนิดข้อมูล} & \textbf{ขนาด (บิต)} & \textbf{คำอธิบาย}\\
\hline
\texttt{byte} & 8 & จำนวนเต็มขนาดเล็ก\\
\texttt{short} & 16 & จำนวนเต็มขนาดกลาง\\
\texttt{int} & 32 & จำนวนเต็มทั่วไป\\
\texttt{long} & 64 & จำนวนเต็มขนาดใหญ่\\
\texttt{float} & 32 & จำนวนทศนิยม\\
\texttt{double} & 64 & จำนวนทศนิยมที่มีความแม่นยำสูง\\
\texttt{char} & 16 & ตัวอักษร Unicode\\
\texttt{boolean} & 1 & ค่าจริงหรือเท็จ (\texttt{true/false})\\
\end{tabular}
\end{center}

\noindent\rule{\textwidth}{0.5pt}
\subsection{นิพจน์และการกำหนดค่าให้ตัวแปร (Expressions \& Variable Assignment)}
\label{sec:org7fa956f}
\begin{itemize}
\item การกำหนดค่าให้ตัวแปรใช้เครื่องหมาย \texttt{=}

\begin{verbatim}
  int x = 10;
  double interest = principal * rate;
\end{verbatim}
\end{itemize}

\noindent\rule{\textwidth}{0.5pt}
\subsection{เมธอด (Subroutines)}
\label{sec:orga5b5ce1}
\begin{itemize}
\item \textbf{Subroutine} คือ ชุดคำสั่งที่รวบรวมไว้ภายใต้ชื่อเดียว และสามารถเรียกใช้ซ้ำได้

\begin{verbatim}
  public static int add(int a, int b) {
      return a + b;
  }
\end{verbatim}
\end{itemize}

\noindent\rule{\textwidth}{0.5pt}
\subsection{สายอักขระ (Strings) **}
\label{sec:orgcf2b760}
\begin{itemize}
\item \textbf{String} คือ ลำดับของตัวอักษร ซึ่งจัดการโดยคลาส \texttt{String}
\item \textbf{เมธอดที่ใช้บ่อย}:
\begin{itemize}
\item \texttt{length()}: คืนค่าความยาวของสายอักขระ
\item \texttt{charAt(index)}: คืนค่าตัวอักษรที่ตำแหน่งที่ระบุ
\item \texttt{toUpperCase()}, \texttt{toLowerCase()}: แปลงสายอักขระเป็นตัวพิมพ์ใหญ่หรือตัวพิมพ์เล็ก
\end{itemize}
\end{itemize}

\noindent\rule{\textwidth}{0.5pt}
\subsection{การใช้ Enum (Enumerations) \textbf{*}}
\label{sec:org23d3081}
\begin{itemize}
\item \textbf{Enum} คือ ชนิดข้อมูลที่มีค่าคงที่ที่กำหนดไว้ล่วงหน้า

\begin{verbatim}
  enum Season { SPRING, SUMMER, FALL, WINTER }
  Season vacation = Season.SUMMER;
\end{verbatim}

\item \textbf{เมธอด \texttt{ordinal()}}: ใช้คืนค่าลำดับของค่าใน Enum

\begin{verbatim}
  System.out.println(Season.SUMMER.ordinal()); // ผลลัพธ์: 1
\end{verbatim}
\end{itemize}

\noindent\rule{\textwidth}{0.5pt}
\subsection{ตัวอย่างการใช้ Enum \textbf{*}}
\label{sec:org1f70104}
\begin{verbatim}
public class EnumDemo {
    enum Day { SUNDAY, MONDAY, TUESDAY, WEDNESDAY, THURSDAY, FRIDAY, SATURDAY }
    public static void main(String[] args) {
        Day tgif = Day.FRIDAY;
        System.out.println(tgif + " is the " + tgif.ordinal() + "-th day of the week.");
    }
}
\end{verbatim}

\noindent\rule{\textwidth}{0.5pt}
\subsection{กฎการตั้งชื่อใน Java (Syntax Rules)}
\label{sec:orge6d20f9}
\begin{itemize}
\item \textbf{Identifiers (ตัวระบุ)} คือ ชื่อที่ใช้เรียกตัวแปร คลาส หรือเมธอด
\item ต้องเริ่มต้นด้วยตัวอักษรหรือ \texttt{\_} และไม่มีช่องว่างระหว่างชื่อ
\item ตัวพิมพ์ใหญ่และตัวพิมพ์เล็กถือว่าแตกต่างกัน เช่น \texttt{HelloWorld} ไม่เหมือนกับ
\texttt{helloworld}
\item ห้ามใช้ \textbf{Reserved Words (คำสงวน)} เช่น \texttt{class}, \texttt{public}, \texttt{static}, \texttt{if},
\texttt{else}
\end{itemize}

\noindent\rule{\textwidth}{0.5pt}
\subsection{โครงสร้างโปรแกรม Java}
\label{sec:orga4b835e}
\begin{itemize}
\item โครงสร้างพื้นฐานของโปรแกรม Java:

\begin{verbatim}
  public class ProgramName {
      public static void main(String[] args) {
          // คำสั่งที่ต้องการให้โปรแกรมทำงาน
      }
  }
\end{verbatim}

\item ชื่อคลาสต้องตรงกับชื่อไฟล์ เช่น คลาส \texttt{HelloWorld} ต้องบันทึกในไฟล์
\texttt{HelloWorld.java}
\end{itemize}

\noindent\rule{\textwidth}{0.5pt}
\subsection{ชนิดข้อมูล Math และเมธอดที่สำคัญ *}
\label{sec:org868097c}
\begin{itemize}
\item คลาส \textbf{Math} มีเมธอดที่ใช้คำนวณต่างๆ เช่น
\begin{itemize}
\item \texttt{Math.abs(x)}: ค่าสัมบูรณ์
\item \texttt{Math.pow(x, y)}: ยกกำลัง
\item \texttt{Math.random()}: สุ่มตัวเลขระหว่าง 0 ถึง 1
\end{itemize}
\end{itemize}

\noindent\rule{\textwidth}{0.5pt}
\subsection{การวัดเวลาในโปรแกรม *}
\label{sec:org043c231}
\begin{itemize}
\item ใช้เมธอด \texttt{System.currentTimeMillis()} เพื่อวัดเวลาปัจจุบันในหน่วยมิลลิวินาที

\item สามารถนำไปใช้วัดเวลาการทำงานของโปรแกรมได้

\begin{verbatim}
  long startTime = System.currentTimeMillis();
  // โค้ดที่ต้องการวัดเวลา
  long endTime = System.currentTimeMillis();
  System.out.println("Run time: " + (endTime - startTime) + " ms");
\end{verbatim}
\end{itemize}

\noindent\rule{\textwidth}{0.5pt}
\subsection{การรับค่า Program Arguments}
\label{sec:org43cca2b}
\begin{itemize}
\item โปรแกรม Java สามารถรับค่า argument จากคอมมานด์ไลน์ได้

\begin{verbatim}
  public static void main(String[] args) {
      System.out.println(args[0]); // แสดงค่า argument ตัวแรก
  }
\end{verbatim}

\item ใช้ \texttt{Integer.parseInt()} หรือ \texttt{Double.parseDouble()} เพื่อแปลงค่าจาก String
เป็นตัวเลข
\end{itemize}

\noindent\rule{\textwidth}{0.5pt}
\subsection{ตัวอย่างโปรแกรม Java}
\label{sec:org597d7ac}
\begin{itemize}
\item \textbf{โปรแกรม HelloWorld}
\end{itemize}

\begin{verbatim}
public class HelloWorld {
    public static void main(String[] args) {
        System.out.println("Hello World!");
    }
}
\end{verbatim}

\begin{itemize}
\item \textbf{โปรแกรมคำนวณดอกเบี้ย (Interest.java)}
\end{itemize}

\begin{verbatim}
public class Interest {
    public static void main(String[] args) {
        double principal = 17000;
        double rate = 0.07;
        double interest = principal * rate;
        principal = principal + interest;

        System.out.println("The interest earned is $" + interest);
        System.out.println("The value after one year is $" + principal);
    }
}
\end{verbatim}

\noindent\rule{\textwidth}{0.5pt}
\end{document}
